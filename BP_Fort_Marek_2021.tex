% arara: xelatex
% arara: xelatex
% arara: xelatex


% options:
% thesis=B bachelor's thesis
% thesis=M master's thesis
% czech thesis in Czech language
% english thesis in English language
% hidelinks remove colour boxes around hyperlinks

\documentclass[thesis=B,english]{FITthesis}[2019/12/23]

\usepackage[utf8]{inputenc} % LaTeX source encoded as UTF-8

% \usepackage{subfig} %subfigures
% \usepackage{amsmath} %advanced maths
% \usepackage{amssymb} %additional math symbols

\usepackage{dirtree} %directory tree visualisation

\usepackage{hyperref} %hypertext links

% % list of acronyms
% \usepackage[acronym,nonumberlist,toc,numberedsection=autolabel]{glossaries}
% \iflanguage{czech}{\renewcommand*{\acronymname}{Seznam pou{\v z}it{\' y}ch zkratek}}{}
% \makeglossaries

\department{Department of \ldots (SPECIFY)}
\title{Finite Automata iPad Editor}
\authorGN{Your Given Name(s) (SPECIFY)} %author's given name/names
\authorFN{Your Family Name (surname, SPECIFY)} %author's surname
\author{Your Name (SPECIFY)} %author's name without academic degrees
\authorWithDegrees{Marek Fořt} %author's name with academic degrees
\supervisor{Your Supervisor's Name (SPECIFY)}
\acknowledgements{THANKS (remove entirely in case you do not with to thank anyone)}
\abstractEN{Summarize the contents and contribution of your work in a few sentences in English language.}
\abstractCS{V n{\v e}kolika v{\v e}t{\' a}ch shr{\v n}te obsah a p{\v r}{\' i}nos t{\' e}to pr{\' a}ce v {\v c}esk{\' e}m jazyce.}
\placeForDeclarationOfAuthenticity{Prague}
\keywordsCS{Replace with comma-separated list of keywords in Czech.}
\keywordsEN{iPad application,  finite automata,  interactive editor,  AlgorithmsLibrary Toolkit, Composable Architecture, Swift}
\declarationOfAuthenticityOption{1} %select as appropriate, according to the desired license (integer 1-6)
% \website{http://site.example/thesis} %optional thesis URL


\begin{document}

% \newacronym{CVUT}{{\v C}VUT}{{\v C}esk{\' e} vysok{\' e} u{\v c}en{\' i} technick{\' e} v Praze}
% \newacronym{FIT}{FIT}{Fakulta informa{\v c}n{\' i}ch technologi{\' i}}

\setsecnumdepth{part}
\setsecnumdepth{section}
\chapter{Introduction}

\section{Motivation, Focus of Thesis}

Theory of finite automata is an important part of computer science curriculum at FIT CTU in Prague and other universities around the world. And although there is a lot of resources one can learn from, there is a lack of those that utilize modern tools. One of such modern tools is iPad (and touch devices in general). This thesis will fill in this gap and build a finite automata editor native application for iPad in this thesis.

Furthermore, I will expand on the recent work done at FIT CTU in Prague concerning development of algorithms library and, more importantly for this thesis, finite automata algorithms including simulating input. This library is named \href{https://alt.fit.cvut.cz/}{Algorithms Library Toolkit} (ALT) \cite{alt} and it has been open sourced.

The main motivation of this thesis is to improve how students learn finite automata and more specifically, enhance the current course BI-AAG that is taught at FIT CTU in Prague. It is also an opportunity to try out algorithms library in practice and create a concrete example of how it can be leveraged.

\section{Thesis Goals}

The main goal of this thesis is to implement a prototype of an automata editor for iPad. This application should enable users to create and edit finite automata with emphasis on touch-based input.
I will also study the web interface of the algorithms library \cite{web-alt}.

\begin{itemize}
	\item prepare and explore how to create interface for automata library
	\item find possible solutions of how to detect automata elements from shapes made by hand on the touch screen
	\item implement a prototype of an automata editor for iPad
	\item assess the usability of SwiftUI and Composable Architecture
	\item perform user testing of the implemented automata editor prototype
\end{itemize}

\section{Thesis Structure}

Let me now introduce you to the structure of the rest of the thesis:

\begin{itemize}
\item In \textbf{Chapter 2} I will go over the theoretical concepts to properly explain terms and concepts on which it will be built upon later.

\item \textbf{Chapter 3} is concerned with analysis of already existing solutions of creating automata editor.

\item \textbf{Chapter 4} is about the design of the editor itself.

\item In \textbf{Chapter 5} I will write about the implementation.

\item \textbf{Chapter 6} will go into the specifics of user testing and its outcomes.

\item In \textbf{Chapter 7} I will assess the usability of the new SwiftUI framework alongside with a functional Composable Architecture.

\item \textbf{Conclusion} is the last chapter of this thesis.

\end{itemize}

\section{Literary Research}

\textbf{This will not be part of the final version of the thesis}

In the literary research I will need to study the current literature for the theoretical background of finite automata. In \autoref{chap:theory} I will use the literature to explain the concepts and terms that are necessary to understand the rest of the thesis. 

For \autoref{chap:analysis} I will need to go through current implementations of automata editors, assessing what value they provide and making sure my thesis is, indeed, creating something new. This can also prove to be helpful during the implementation as I can be inspired by projects that have already been made.

\autoref{chap:design} is concerned with design of the editor - the literature will be focused on the UX and UI of interactive editors, not limited to finite automata. Besides that, I will also study available methods for discerning automata shapes drawn by hand and going through already available technology on this topic. Last but not least, I will go over the architectural design in the form of Composable Architecture - this will encompass studying other functional and non-functional architectures. This part of research will also include research on SwiftUI and how it compares with UIKit (UI framework predecessor) and other declarative libraries such as React for web. 


\setsecnumdepth{all}
\chapter{Theory}
\label{chap:theory}

\chapter{Analysis}
\label{chap:analysis}

\chapter{Automata Editor Design}
\label{chap:design}

\chapter{Implementation}
\label{chap:implementation}

\chapter{User testing}

\chapter{SwiftUI and Composable Architecture Assessment}

\chapter{Conclusion}

\section{Goals Assessment}

One of the goals of this thesis was to assess the usability of new iOS programming paradigm of SwiftUI with the new functional architecture pattern Composable Architecture. This combination has proven to be beneficial and it is a good option for new iOS applications.

The main goal of this project was to create a prototype of automata editor for iPad while showcasing the Algorithms Library Toolkit that is being continuously developed at FIT CTU.

I have successfully implemented editing and simulating input of finite automata while creating a good example of capabalities of Algorithms Library Toolkit.

One of the other requirements of the automata editor has been to discern automata elements from shapes drawn by hand on the device. The solution chosen during the analysis made this possible as has been described in more detail in \autoref{chap:implementation}.

\section{Thesis Contribution}

The contribution of this thesis is testing Algorithms Library Toolkit in practice and can now be pointed to for users who want to see its capabilities. The editor can now also be recommended in the course of BI-AAG at FIT CTU (and other universities) - for students and teachers alike.

\section{Future Work}

Algorithms Library Toolkit is an extensive library and there are still capabilites that are not implemented in the editor.

The choice of developing a native iOS application has resulted in good UX, but in the future it would be beneficial to broaden the possible audience and either develop a similar Android app or create a more ubiquituous web interface.

\setsecnumdepth{part}

\setsecnumdepth{all}
\appendix

\chapter{Acronyms}
% \printglossaries
\begin{description}
	\item[GUI] Graphical user interface 
	\item[UX] User experience 
\end{description}


\chapter{Contents of enclosed CD}

%change appropriately

\begin{figure}
	\dirtree{%
		.1 readme.txt\DTcomment{the file with CD contents description}.
		.1 src\DTcomment{the directory of source codes}.
		.2 editor\DTcomment{implementation sources}.
		.2 thesis\DTcomment{the directory of \LaTeX{} source codes of the thesis}.
		.1 text\DTcomment{the thesis text directory}.
		.2 thesis.pdf\DTcomment{the thesis text in PDF format}.
		.1 usertesting.pdf\DTcomment{the directory of source codes}.
		.1 editor\DTcomment{directory of screenshots from the editor}.
	}
\end{figure}

\bibliography{BP_Fort_Marek_2021}
\bibliographystyle{ieeetr}
\end{document}
