% arara: xelatex
% arara: xelatex
% arara: xelatex


% options:
% thesis=B bachelor's thesis
% thesis=M master's thesis
% czech thesis in Czech language
% english thesis in English language
% hidelinks remove colour boxes around hyperlinks

\documentclass[thesis=B,english]{FITthesis}[2019/12/23]

%\usepackage[utf8]{inputenc} % LaTeX source encoded as UTF-8
% \usepackage[latin2]{inputenc} % LaTeX source encoded as ISO-8859-2
% \usepackage[cp1250]{inputenc} % LaTeX source encoded as Windows-1250

% \usepackage{subfig} %subfigures
% \usepackage{amsmath} %advanced maths
% \usepackage{amssymb} %additional math symbols

\usepackage{dirtree} %directory tree visualisation

% % list of acronyms
% \usepackage[acronym,nonumberlist,toc,numberedsection=autolabel]{glossaries}
% \iflanguage{czech}{\renewcommand*{\acronymname}{Seznam pou{\v z}it{\' y}ch zkratek}}{}
% \makeglossaries

% % % % % % % % % % % % % % % % % % % % % % % % % % % % % % 
% EDIT THIS
% % % % % % % % % % % % % % % % % % % % % % % % % % % % % % 

\department{Department of \ldots (SPECIFY)}
\title{Thesis title (SPECIFY)}
\authorGN{Your Given Name(s) (SPECIFY)} %author's given name/names
\authorFN{Your Family Name (surname, SPECIFY)} %author's surname
\author{Your Name (SPECIFY)} %author's name without academic degrees
\authorWithDegrees{Your Name \& degrees} %author's name with academic degrees
\supervisor{Your Supervisor's Name (SPECIFY)}
\acknowledgements{THANKS (remove entirely in case you do not with to thank anyone)}
\abstractEN{Summarize the contents and contribution of your work in a few sentences in English language.}
\abstractCS{V n{\v e}kolika v{\v e}t{\' a}ch shr{\v n}te obsah a p{\v r}{\' i}nos t{\' e}to pr{\' a}ce v {\v c}esk{\' e}m jazyce.}
\placeForDeclarationOfAuthenticity{Prague}
\keywordsCS{Replace with comma-separated list of keywords in Czech.}
\keywordsEN{Replace with comma-separated list of keywords in English.}
\declarationOfAuthenticityOption{1} %select as appropriate, according to the desired license (integer 1-6)
% \website{http://site.example/thesis} %optional thesis URL


\begin{document}

% \newacronym{CVUT}{{\v C}VUT}{{\v C}esk{\' e} vysok{\' e} u{\v c}en{\' i} technick{\' e} v Praze}
% \newacronym{FIT}{FIT}{Fakulta informa{\v c}n{\' i}ch technologi{\' i}}

\setsecnumdepth{part}
\setsecnumdepth{section}
\chapter{Introduction}

\section{Motivation, Focus of Thesis}

Finite automata have been first described by Warren McCulloch and Walter Pitts in 1943 and since then they have become one of the cornerstones of computer science. While finite automata research is limited nowadays, it is still something that we build upon today and every student of computer science needs to understand its concepts.

And although there is a lot of resources one can learn from, there is a lack of those that utilize modern tools. One of such modern tools is iPad (and touch devices in general). I'd like to fill in this gap and build a finite automata editor native application for iPad in this thesis.

Furthermore, I'd like to expand on the recent work done at FIT CTU concerning development of algorithms library and, more importantly for this thesis, finite automata algorithms including simulating input. This library is named Algorithms Library Toolkit and it has been open sourced.

Last but not least, in this thesis I will try out the new iOS programming declarative paradigm which has been introduced by Apple with the new SwiftUI framework with the combination of a functional architecture, called Composable Architecture.

The main motivation of this thesis is to improve how students learn finite automata and more specifically, enhance the current course BI-AAG that is taught at FIT CTU. It's also an opportunity to try out algorithms library in practice and create a concrete example of how it can be leveraged.

\section{Thesis Goals}

Goals of this thesis are the folowing:

\begin{itemize}
	\item prepare and explore how to create interface for automata library
	\item find possible solutions of how to detect automata elements from shapes made by hand on the touch screen
	\item implement a prototype of an automata editor for iPad
	\item assess the usability of SwiftUI and Composable Architecture
	\item perform user testing of the implemented automata editor prototype
\end{itemize}

\section{Thesis Structure}

Let me now introduce you to the structure of the rest of the thesis:

\begin{itemize}
\item In \textbf{Chapter 2} I will go over the theoretical concepts to properly explain terms and concepts on which it will be built upon later.

\item \textbf{Chapter 3} is concerned with analysis of already existing solutions of creating automata editor.

\item \textbf{Chapter 4} is about the design of the editor itself.

\item In \textbf{Chapter 5} I will write about the implementation.

\item \textbf{Chapter 6} will go into the specifics of user testing and its outcomes.

\item In \textbf{Chapter 7} I will assess the usability of the new SwiftUI framework alongside with a functional Composable Architecture.

\item \textbf{Conclusion} is the last chapter of this thesis.

\end{itemize}

\setsecnumdepth{all}
\chapter{Theory}

\chapter{Analysis}

\chapter{Automata Editor Design}

\chapter{Implementation}

\chapter{User testing}

\chapter{SwiftUI and Composable Architecture Assessment}

\chapter{Conclusion}

\setsecnumdepth{part}

\bibliographystyle{iso690}
\bibliography{mybibliographyfile}

\setsecnumdepth{all}
\appendix

\chapter{Acronyms}
% \printglossaries
\begin{description}
	\item[GUI] Graphical user interface
	\item[XML] Extensible markup language
\end{description}


\chapter{Contents of enclosed CD}

%change appropriately

\begin{figure}
	\dirtree{%
		.1 readme.txt\DTcomment{the file with CD contents description}.
		.1 exe\DTcomment{the directory with executables}.
		.1 src\DTcomment{the directory of source codes}.
		.2 wbdcm\DTcomment{implementation sources}.
		.2 thesis\DTcomment{the directory of \LaTeX{} source codes of the thesis}.
		.1 text\DTcomment{the thesis text directory}.
		.2 thesis.pdf\DTcomment{the thesis text in PDF format}.
		.2 thesis.ps\DTcomment{the thesis text in PS format}.
	}
\end{figure}

\end{document}
