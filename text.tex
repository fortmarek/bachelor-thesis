%---------------------------------------------------------------
\chapter{Introduction}
%---------------------------------------------------------------
\setcounter{page}{1}

\section{Motivation, Focus of Thesis}

Theory of finite automata is an important part of computer science curriculum at FIT CTU in Prague and other universities around the world. And although there is a lot of resources one can learn from, there is a lack of those that utilize modern tools. One of such modern tools is iPad (and touch devices in general). This thesis will fill in this gap and build a finite automata editor native application for iPad in this thesis.

Furthermore, I will expand on the recent work done at FIT CTU in Prague concerning development of algorithms library and, more importantly for this thesis, finite automata algorithms including simulating input. This library is named \href{https://alt.fit.cvut.cz/}{Algorithms Library Toolkit} (ALT) \cite{alt} and it has been open sourced.

The main motivation of this thesis is to improve how students learn finite automata and more specifically, enhance the current course BI-AAG that is taught at FIT CTU in Prague. It is also an opportunity to try out algorithms library in practice and create a concrete example of how it can be leveraged.

\section{Thesis Goals}

The main goal of this thesis is to implement a prototype of an automata editor for iPad. This application should enable users to create and edit finite automata with emphasis on touch-based input.
I will also study the web interface of ALT \cite{web-alt}, ALT itself focusing on design and drawing of finite automata, the possibilities of strokes detection on touch devices, and the approaches of shape detection, especially those used in automata drawing on iOS platform.

After the initial study of current approaches, I will implement a prototype of a finite automata editor iPad app that will be capable of finite automata recognition from strokes.

I will then conduct usability testing to assess the usability and shortcomings of the prototype.

\section{Thesis Structure}

Let me now introduce you to the structure of the rest of the thesis:

\begin{itemize}
\item In \textbf{Chapter 2} I will go over the theoretical concepts to properly explain terms and concepts on which it will be built upon later.

\item \textbf{Chapter 3} is concerned with analysis of already existing solutions of creating automata editor, the existing ALT web interface and ALT itself.

\item \textbf{Chapter 4} is about the design of the editor itself.

\item In \textbf{Chapter 5} I will write about the implementation.

\item \textbf{Chapter 6} will go into the specifics of user testing and its outcomes.

\item \textbf{Conclusion} is the last chapter of this thesis.

\end{itemize}

\chapter{Theory}
\label{chap:theory}

Firstly,  I will need to define terms and formal definitions concerning mainly finite automata theory, as that is the main subject of this thesis, and machine learning (maybe!!)

\section{Formal Languages and Grammars}
 
The following definitions are taken from Automata and Grammars by Eliška Šestáková \cite{automata-and-grammars}, Introduction to Automata Theory, Languages, and Computation \cite{introduction-automata}, and materials from BIE-AAG course \cite{lectures}.

\subsection{Formal Languages}

\begin{definition}
Alphabet (conventionally denoted by $\Sigma$) is a finite set whose elements are called symbols.
\end{definition}
Alphabets therefore can be:
\begin{itemize}
    \item $\Sigma$ = \{0, 1\}
    \item $\Sigma$ = \{a, b, c, d, e\}
    \item $\Sigma$ = \{one, two\}
\end{itemize}

\begin{definition}
String (word) over an alphabet is a finite sequence of symbols from that alphabet.
\end{definition}
\begin{itemize}
    \item $\epsilon$ - \textit{empty string} (string with zero occurences of symbols)
    \item $\Sigma^{*}$ - set of all strings over $\Sigma$
    \item $\Sigma^{+}$ - set of all nonempty strings over $\Sigma$
\end{itemize}
For a binary alphabet $\Sigma$ = \{0, 1\} $\epsilon$, 1001, 100, 1, 001 are all strings over the alphabet $\Sigma$.

\begin{definition}
Formal language L over an alphabet $\Sigma$ is any subset of all the strings over $\Sigma$ - i.e., $L \subseteq \Sigma$
\end{definition}
For a binary alphabet $\Sigma$ = \{0, 1\} a formal language over $\Sigma$ are then subsets of \textit{all} binary strings. We can denote the language either by:
\begin{itemize}
    \item enumeration notation where all possible strings in the language are listed, e.g.: $\textit{L}_1 = \{\epsilon\}, \textit{L}_2 = \{1\}, \textit{L}_3 = \{0, 00, 000, 01\}$ 
    \item set-builder notation where the languages are described in the following way: \{ w $\mid$ something about w \}. Examples are: $\textit{L}_4 = \{\textit{w} \mid \textit{w} \in {0,1}^* \wedge \left|\textit{w}\right| mod 2 = 0\}$, $\textit{L}_5 = \{0^\textit{n}1^\textit{n} : \textit{n} \in \mathbb{N}_0\}$
\end{itemize}

\subsection{Grammar}

Grammars are used to describe languages. Below you can find how they are defined:

\begin{definition}
    \textit{Grammar} is a quadruple of G = (N, $\Sigma$, P, S) where:
\end{definition}
\begin{itemize}
    \item \textit{N} is a finite non-empty set of nonterminal symbols
    \item $\Sigma$ is a finite set of terminal symbols ($\Sigma$ $\cap$ \textit{N} = $\emptyset$). Note that \textit{N} $\cap$ $\Sigma$ = $\emptyset$
    \item \textit{P} is a finite set of \textit{production rules}, assuming the following form:\\

    \centerline{$\alpha\textit{A}\beta \rightarrow \gamma$ ($\alpha, \beta, \gamma \in (\textit{N} \cup \Sigma)^*$)}
\end{itemize}

The following is an example of a grammar that describes the language \textit{L} = $\{01^\textit{n}0 : \textit{n} \in \mathbb{N}_0\}$:
Grammar \textit{G} = (\{\textit{A}, \textit{S}\}, \{0, 1\}, \textit{P}, \textit{S}) where \textit{P}:
\begin{itemize}
    \item \textit{S} $\rightarrow$ 0\textit{A}
    \item \textit{A} $\rightarrow$ 1\textit{A}
    \item \textit{A} $\rightarrow$ 0
\end{itemize}

\subsection{Chomsky Classification of Grammars}

Grammars are divided into four classes where they differ in their production rules. 

\begin{definition}
    Let G = (N, $\Sigma$, P, S). We say that G is:
\end{definition}
\begin{enumerate}
    \item \textit{Unrestricted grammar} (type 0), if every rule is in the form of:
    
    \centerline{$\alpha\textit{A}\beta \rightarrow \gamma$ ($\alpha, \beta, \gamma \in (\textit{N} \cup \Sigma)^*$, $\textit{A} \in \textit{N}$)}
    \item \textit{Context-sensitive} (type 1), if every rule is in the form of:

    \centerline{$\gamma\textit{A}\delta \rightarrow \gamma\alpha\delta$ ($\gamma, \delta \in (\textit{N} \cup \Sigma)^*$, $\textit{a} \in (\textit{N} \cup \Sigma)^+$)}
    or in the form of $\textit{S} \rightarrow \epsilon$ if \textit{S} is not present on the right hand side of any rule of a given grammar.
    \item \textit{Context-free grammar} (type 2) if every rule is in the form of:

    \centerline{$\textit{A} \rightarrow \alpha$ $(\textit{A} \in \textit{N}, \alpha \in (N \cup \Sigma)^*)$}
    \item \textit{Regular grammar} (type 3), if every rule is in the form of:

    \centerline{$\textit{A} \rightarrow \textit{a}$ or $\textit{A} \rightarrow \textit{aB}$ $(\textit{a} \in \Sigma, \textit{A}, \textit{B} \in \textit{N})$}
    or in the form of $\textit{S} \rightarrow \epsilon$ if \textit{S} is not present on the right hand side of any rule of a given grammar.
\end{enumerate}

\subsection{Classification of Languages}
Classification of languages, also known as the Chomsky hierarchy, has the following definition:
\begin{definition}
    We say that language is:
\end{definition}
\begin{enumerate}
    \item \textit{formal} if it is a formal language but is neither regular, context-free, context-sensitive, nor recursive enumerable. These languages are not recognized by Turing machine.
    \item \textit{recursively enumerable} if and only if $\exists$ unrestricted grammar which generates it
    \begin{itemize}
        \item recognized by Turing machine
    \end{itemize}
    \item \textit{context-sensitive} if and only if $\exists$ context-sensitive grammar which generates it
    \begin{itemize}
        \item recognized by linear bounded Turing machine
    \end{itemize}
    \item \textit{context-free} if and only if $\exists$ context-free grammar which generates it
    \begin{itemize}
        \item recognized by a nondeterministic pushdown automaton
    \end{itemize}
    \item \textit{regular} if and only if $\exists$ regular grammar that generates it
    \begin{itemize}
        \item recognized by finite automaton
    \end{itemize}
\end{enumerate}

In this thesis we will be mainly interested in regular languages / grammars since those are recognized by finite automata. Finite automata will be defined in the following section.

\section{Finite Automata}

The final editor app will be for finite automata, therefore they are very important for this thesis. Informally, a finite automaton is a model for simple computation. States, that serve as memory, and transitions together form a \textit{control unit}. Along with a control unit, the finite automaton has a \textit{read-only input tape}, which is divided into individual cells, and the \textit{head} that scans the input tape as the automaton continuously reads it, cell by cell. Automaton starts in its initial state and with head pointing at the first cell. As the input is read, the head moves until it has read all of the input tape. If there is a missing transition for an input, the automaton does not accept the input. Otherwise, it accepts the input if it is in an end state at the end of the input.

Let's define a finite automaton formally:
\begin{definition}
    Finite automaton is a quintuple \textit{M} = (\textit{Q}, $\Sigma$, $\delta$, $\textit{q}_0$, \textit{F}) where:
    \begin{itemize}
        \item \textit{Q} is a finite non-empty set of states
        \item $\Sigma$ is a finite input alphabet
        \item $\delta$ is the transition function (the exact definition is determined by which type of finite automaton it is - see below)
        \item $\textit{q}_0 \in \textit{Q}$ is the initial state
        \item $\textit{F} \subseteq \textit{Q}$ is the set of initial states
    \end{itemize}
\end{definition}

Finite automaton can also be either \textit{deterministic} finite automaton or \textit{nondeterministic} finite automaton. This dictates the exact definition of $\delta$ - transition function.
For deterministic finite automaton (DFA) the definition of $\delta$ is:\\
\centerline{$\delta$ is a mapping from \textit{Q} $\times$ $\Sigma$ to \textit{Q}}
$\delta$ for nondeterministic finite automaton (NFA) is defined as:\\
\centerline{$\delta$ is a mappping from \textit{Q} $\times$ $\Sigma$ into the set of all subsets \textit{Q} (denoted by $2^\textit{Q}$)}
Expanding upon the difference between the definition of the transition function:
\begin{itemize}
    \item DFA can only transition from one state to another, e.g. from $\textit{q}_0$ to $\textit{q}_1$ ($\textit{q}_0$, $\textit{q}_1$ $\in$ \textit{Q})
    \item NFA can transition to a set of states, e.g. from $\textit{q}_0$ to $\textit{q}_1$ ($\textit{q}_0$, $\textit{q}_1$, $\textit{q}_2$ $\in$ \textit{Q})
\end{itemize}
If we change the definition of NFA's $\delta$ to a mapping from \textit{Q} $\times$ ($\Sigma \cup \{\epsilon\}$) we allow, what are called, $\epsilon$-transitions that allow us to move to a different state while not reading any input from the tape. This finite automaton is then called \textit{nondeterministic finite automaton with $\epsilon-transitions$}.

\chapter{Analysis}
\label{chap:analysis}

\chapter{Automata Editor Design}
\label{chap:design}

\chapter{Implementation}
\label{chap:implementation}

\chapter{User Testing}

\chapter{Conclusion}

\section{Goals Assessment}

Assess my goals.

\section{Thesis Contribution}

The contribution of this thesis is testing Algorithms Library Toolkit in practice and can now be pointed to for users who want to see its capabilities. The editor can now also be recommended in the course of BI-AAG at FIT CTU (and other universities) - for students and teachers alike.

\section{Future Work}

Algorithms Library Toolkit is an extensive library and there are still capabilites that are not implemented in the editor.

The choice of developing a native iOS application has resulted in good UX, but in the future it would be beneficial to broaden the possible audience and either develop a similar Android app or create a more ubiquituous web interface.
